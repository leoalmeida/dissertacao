\chapter{Arquitetura proposta e Padr\~{o}es Utilizados}
\section{Arquitetura}
\subsection{sub-item x}

\section{Publisher/Subscriber}
\subsection{sub-item x}

\section{Big Data}
De forma a complementar o trabalho realizado pelas plataformas de IoT, mais focadas na cria��o do ecossistema de comunica��o entre n�s da Internet das Coisas, � primordial a implanta��o em segundo plano de estruturas complexas respons�veis pelo processamento dos dados gerados dentro do ecossistema. S� assim conseguimos extrair a intelig�ncia computacional, nos aproximando dos benef�cios propostos pela Internet das Coisas.

Essa estrutura, utilizando uma ou mais ferramentas, recebe dados provenientes do ecossistema (publica��es, notifica��es, mensagens, acionamentos, etc.) e inicia um fluxo de trabalho percorrendo diversas regras de transfoma��o pr�-programadas. Ao final desse fluxo de trabalho informa��es �teis s�o criadas, que podem ser utilizadas tanto na tomada de decis�o quanto para reinser��o na estrutura, criando assim um ciclo de intelig�ncia computacional. Dentre algumas das caracter�sticas importantes a se avaliar nessas ferramentas temos: 

\begin{itemize}
	\item Engines de processamento em lote.
	\item S�o baseadas em clusters.
	\item Capacidade de processamento distribu�do e paralelo
	\item Usa a mem�ria principal intensivamente.
	\item Tem como requisitos ess�nciais a baixa lat�ncia e larga escalabilidade.
	\item Acesso de alto rendimento aos dados 
\end{itemize}

Com base nos conceitos avaliados sobre Internet das Coisas e Big Data, selecionamos algumas das ferramentas open source que prop�em boas solu��es para an�lise de dados em tempo real. Por�m, como a forma de implementa��o dessas ferramentas varia bastante, assim como as necessidades de projeto, uma ferramenta pode se mostrar mais adequada que outra dependendo da sua utiliza��o. Abordaremos algumas dessas caracter�sticas espec�ficas nos t�picos abaixo.






