\chapter{Introdu\c{c}\~{a}o/motiva\c{c}\~{a}o}
Muitas vezes ao pensar em internet das coisas (IoT), contextualizando seu impacto na intera\c{c}\~{a}o entre pessoas e o ambiente ao seu redor, \'{e} enfatizado o uso de informa\c{c}\~{o}es traduzidas do ambiente (temperatura, humidade e etc.) para automatiza\c{c}\~{a}o, controle ou qualquer tomada de decis\~{a}o. Contudo, a inclus\~{a}o de informa\c{c}\~{o}es proveniente das redes sociais, estruturando um senso comum, se torna um fator de grande impot�ncia na forma\c{c}\~{a}o de uma opini\~{a}o acertiva em rela\c{c}\~{a}o aos acontecimentos perif\'{e}ricos ao nosso foco principal.

O objetivo principal desse estudo \'{e} estruturar o conceito de Internet das Coisas e suas deriva\c{c}\~{o}es, cujo primeiro passo para uma efetiva implanta\c{c}\~{a}o \'{e} criar nas pessoas envolvidas um bom entedimento sobre seus conceitos. Contudo, em uma breve revis\~{a}o da literatura existente no mundo podemos verificar o qu\~{a}o controverso s\~{a}o esses conceitos. A proposta desse documento \'{e} apresentar alguns fatos desde a origem do conceito at\'{e} os dias atuais de forma a facilitar o entendimento do assunto. Sob um diferente ponto de vista, pretendo demonstrar a cont\'{i}gua rela\c{c}\~{a}o de conceitos da Internet das Coisas com outros termos como Computa\c{c}\~{a}o Ub\'{i}qua, Disappearing Computer e Web das Coisas e Web Social das Coisas.

A segunda parte, busca apresentar brevemente algumas das tecnologias que est\~{a}o fortalecendo o conceito de Internet das Coisas e devem conduzir seu futuro.

Esse artigo busca ajudar no entedimento sobre alguns conceitos b\'{a}sicos sobre Internet das Coisas, a origem desse paradigma, pontuando semelhan\c{c}a com outros conceitos e por fim apresentando algumas das tecnologias que est\~{a}o aparecendo para definir a base do ecossistema da Internet das Coisas e ferramentas importantes no processo de minera\c{c}\~{a}o dos dados provenientes de dispositivos conectados.

	\section{Das origens � Web Social das Coisas}
	Considerando o quanto o termo Internet das Coisas \'{e} utilizado em nosso dia-a-dia de trabalho e pesquisa, juntando ao fato de que esse termo \'{e} bastante controverso na literatura de computa\c{c}\~{a}o, sendo at\'{e} mesmo confundido com outros termos, penso que a melhor forma de iniciarmos a sua compreens\~{a}o seria entendendo seu atual momento e identificando fatos do passado que a levaram aos seus padr\~{o}es atuais de funcionamento. 
	
	Como fatos relevantes, entendo que eventos motivadores de sua cria\c{c}\~{a}o, de seus termos e objetivos s\~{a}o de grande importancia, assim como o fato que conduziu seus idealizadores a escolha dos padr\~{o}es atuais em detrimento de outros existentes e desafios ou necessidades do passado que motivaram mudan\c{c}as nos seus objetivos.
	
	At\'{e} a d\'{e}cada de 60, ocorreram diversos fatos relacionados � cria\c{c}\~{a}o e expan\c{c}\~{a}o das telecomunica\c{c}\~{o}es no mundo, fatos que n\~{a}o s\~{a}o o foco desse artigo. Ap\'{o}s esse per\'{i}odo de evolu\c{c}\~{a}o das comunica\c{c}\~{o}es, que os cientistas come\c{c}aram os debates sobre como seria se grande parte do mundo que vivemos pudesse ser virtualizado, sendo esses debates um press\'{a}gio para a Internet das coisas.
	
		\paragraph{Anos 60}
		Em seu livro "Understanding Media" McLuhan~\cite{McLuhan1964} descrevia que por meio de m\'{i}dias eletr\^{o}nicas, eles haviam configurado uma din�mica onde todas as tecnologias anteriores (incluindo as cidades como conheciam) seriam traduzidas em sistemas de informa\c{c}\~{a}o. Nos anos seguintes Karl Steinbuch~\cite{KlausBiener1997}. No fim da d\'{e}cada foi criada a Arpanet, primeira rede computadores, iniciando o ciclo de desenvolvimento da internet ~\cite{MichaelHAUBEN1995}. 
		
		\paragraph{Anos 80}
		O DNS(Domain Name System) foi criado para facilitar o acesso � internet atrav\'{e}s de dom\'{i}nios no lugar dos n\'{u}meros IP ~\cite{Mockapetris1983}, enquanto que Tim Berners-Lee fez a proposta de cria\c{c}\~{a}o da rede mundial de computadores~\cite{Berners-Lee1989}.
		
		No final desse per\'{i}odo, Mark Weiser(CTO da Xerox PARC) criou o projeto Computa\c{c}\~{a}o Ub\'{i}qua visando responder rapidamente erros do conceito de computador pessoal: alta complexidade; dif\'{i}cil de utilizar; exige demais a aten\c{c}\~{a}o dos usu\'{a}rios; isola muito as pessoas do mundo; e excessivamente dominante, por ocupar muito a vida e mesa de trabalho das pessoas~\cite{Weiser1999}.
		
		\paragraph{Anos 90}
		Com a rede mundial de computadores ainda embrion\'{a}ria, o conceito de Internet das Coisas come\c{c}ou a se formar partindo da torradeira criada por John Romkey, tendo em vista a demonstra\c{c}\~{a}o que ele e seus colegas fariam na confer\^{e}ncia INTEROP para apresentar o protocolo de rede SNMP que estavam criando naquele momento ~\cite{Romkey2017}. Esse se tornou o primeiro dispositivo IoT, que conectada a um computador em rede e usando uma base de informa\c{c}\~{a}o (SNMP MIB), pode ligada e desligada remotamente.
		
		Nesse per\'{i}odo, Mark Weiser com suas publica\c{c}\~{o}es, definiu alguns dos principais conceitos da Internet das Coisas. Inicialmente em seu primeiro artigo sobre o assunto ele define a Computa\c{c}\~{a}o Ub\'{i}qua como tecnologias que desaparecerem, compostas por elementos delas mesmas na ess\^{e}ncia da vida cotidiana at\'{e} se tornarem igualmente impercept\'{i}veis a si pr\'{o}prias~\cite{Weiser1991}, alguns anos depois em outro artigo ele define a computa\c{c}\~{a}o Ub\'{i}qua como o oposto da realidade virtual, onde de um lado as pessoas s\~{a}o conduzidas a um mundo criado por computadores, do outro \'{e} institu\'{i}do �s m\'{a}quinas que coabitem com as pessoas no mundo real ~\cite{Weiser1993}. Em um terceiro artigo, ao descrever um projeto de circuito el\'{e}trico criado por sua colega de empresa, Weiser cunhou "Tecnologias Calmas" ou "Intelig\^{e}ncia ambiental" com a seguinte frase: "N\~{a}o requer nenhum espa\c{c}o na tela do seu computador e de fato n\~{a}o usa ou comp\~{o}es computadores. N\~{a}o utiliza softwares, somente alguns d\'{o}lares em hardware e pode ser compartilhado por v\'{a}rias pessoas ao mesmo tempo." ~\cite{Weiser:1997:CAC:504928.504934}.
		
		Ao criar o 'Trojan Room Coffee Pot' para monitorar a quantidade de caf\'{e} da m\'{a}quina do laborat\'{o}rio de computa\c{c}\~{a}o da Universidade de Cambridge, Quentin Stafford-Fraser e Paul Jardetzky acabaram por criar o que podemos considerar como um dos prim\'{o}rdios do IoT. Se tratava de uma camera que mantinha atualizado nos servidores do pr\'{e}dio, imagens (3 por minuto) da m\'{a}quina de caf\'{e} para que as pessoas pudessem consultar se tinha caf\'{e}, evitando uma viagem perdida ~\cite{Stafford-Fraser1995}. Na mesma \'{e}poca foi criada a WearCam por Steve Mann, considerada o primeiro "Wearable" ~\cite{SteveMann1995}.
		
		Quando Paul Saffo publicou o artigo "Sensors: The Next Wave of Infotech Innovation"~\cite{Saffo:1997:SNW:253671.253734}, ele descreveu os motivos que fariam dos sensores ubiquos a pr\'{o}xima onda de inova\c{c}\~{a}o concluindo que num futuro pr\'{o}ximo, sensores anal\'{o}gicos seriam facilmente interligados � computadores digitais e � redes de computadores criando uma rede de sensores. Comprovando essa tese, o projeto inTouch ~\cite{Brave:1997:IMH:1120212.1120435} criou uma tecnologia chamada de "telefone tang\'{i}vel", que sincroniza objetos f\'{i}sicos de forma distribu\'{i}da utilizando sensores para comunica\c{c}\~{a}o t\'{a}til � longa dist�ncia.
		
		Em 1999, Kevin Ashton criou o termo Internet das Coisas ao descrever um sistema onde a internet \'{e} conectada ao mundo real utilizando diversos sensores ubiquos. Ashton~\cite{Ashton2009} cita como acontenceu, em seu artigo: "Poderia estar errado, mas eu estou certo que o termo Internet das Coisas teve inicio como t\'{i}tulo de uma apresenta\c{c}\~{a}o feita por mim na Procter \& Gamble em 1999. Ao unir a nova ideia de utiliza\c{c}\~{a}o do RFID na cadeia de suprimentos da P\&G aos t\'{o}picos acalorados at\'{e} ent\~{a}o sobre Internet foi uma forma encontrada de chamar a aten\c{c}\~{a}o dos executivos. Isso elucida algo que habitualmente cria um mal entendido."
		
		\paragraph{Anos 2000}
		
		Com a dissemina\c{c}\~{a}o do termo Internet das Coisas, ele passa a ser mencionado inumeras vezes em publica\c{c}\~{o}es convencionais, como The Guardian, Scientific American e The Boston Globe e pela primeira vez come\c{c}a a aparecer nos t\'{i}tulos de livros. Come\c{c}am a aparecer projetos visando a implementa\c{c}\~{a}o de algumas das ideias propostas, como o Cooltown, o Internet0 e a iniciativa "Disappearing Computer".
		
		A tecnologia RFID come\c{c}a a ser implementada em larga escala pelo departamento de defesa americano nos programas de combate � viol\^{e}ncia sexual (Savi Program) e comercialmente pelo Walmart em suas pr\'{o}prias lojas. 
		
		A Internet das Coisas come\c{c}a a se consolidar quando a Uni\~{a}o Europ\'{e}ia, reconhece a import�ncia acad\^{e}mica do assunto sediando sua primeira confer\^{e}ncia internacional e a Uni\~{a}o Internacional de Telecomunica\c{c}\~{o}es (ITU) publica seu primeiro relat\'{o}rio com a seguinte afirma\c{c}\~{a}o: "Uma nova dimens\~{a}o tem sido inclu\'{i}da no mundo das tecnologias da informa\c{c}\~{a}o e comunica\c{c}\~{a}o (ICTs): A qualquer hora, as pessoas podem se conectar em qualquer lugar, agora teremos conectividade em tudo. As conex\~{o}es ir\~{a}o se multiplicar e criar uma din�mica inteiramente nova de rede de redes - uma Internet das Coisas" ~\cite{ITU2005}.
		
		Seu reconhecimento internacional incentivou um grupo de empresas da \'{a}rea de tecnologia, comunica\c{c}\~{o}es e energia fundar a IPSO Alliance para promo\c{c}\~{a}o da utiliza\c{c}\~{a}o do protocolo IP em redes de objetos inteligentes. Al\'{e}m das empresas, muitos governos tamb\'{e}m mostraram grande interesse na nova ind\'{u}stria, principalmente a china ao injetar grande quantidade de recursos nos fundos de pesquisas de suas principais instituti\c{c}\~{o}es ~\cite{Zhengyan2009}. De forma a viabilizar o avan\c{c}o de pesquisas em novas tecnologias, o FCC liberou a utiliza\c{c}\~{a}o dos espa\c{c}os sem uso entre as frequ\^{e}ncias 470 MHz e 698 MHz. 
		
		Com pesquisas sendo realizadas nos 4 cantos do mundo, o Conselho Nacional de intelig\^{e}ncia Americano lista a Internet das Coisas como uma das 6 Tecnologias civ\'{i}s disruptivas com potencial impacto nos interesses Americanos at\'{e} o ano de 2025 e pesquisa citada em ~\cite{Evans2011} mostra que a quantidade de "objetos ou coisas" (em referencia aos dispositivos m\'{o}veis) conectados � internet j\'{a} atingiam 12.5 bilh\~{o}es em 2010, enquanto que a popula\c{c}\~{a}o chegava a 6.8 bilh\~{o}es de pessoas no mundo, al\c{c}ando o patamar de 1,84 dispositivos conectados por pessoa. 
		
		Prevendo a incapacidade do IPv4 em atender a acentuada expans\~{a}o na conex\~{a}o de equipamentos � internet a IETF (Internet Engineering Task Force) lan\c{c}ou ao p\'{u}blico o padr\~{a}o IPV6 ~\cite{WORLDIPV62012} permitindo a conex\~{a}o de aproximadamente 340 undecilh\~{o}es de endere\c{c}os ou como apontado por Steven Leibson, "Podemos assinalar um endere\c{c}o IPV6 para cada \'{a}tomo na superf\'{i}cie da terra e ainda assim teremos endere\c{c}os suficientes para endere\c{c}amento de outros 100 ou mais planetas terra.".
		
		De forma a promover uma abordagem universal no desenvolvimento de padr\~{o}es t\'{e}cnicos, a Uni\~{a}o Internacional de Telecomunica\c{c}\~{o}es cria o grupo de estudos IoT-GSI(depois transformado em SG20), pemitindo um alcance global � Internet das Coisas. No ramo dos neg\'{o}cios, o instituto de pesquisa gartner inclui o termo Internet das Coisas em seu Hype Cycle anual, respons\'{a}vel pelo rastreamento do ciclo de vida das tecnologias da sua apari\c{c}\~{a}o ao seu plat\^{o} de produtividade. A Internet das Coisas atinge seu pico de expectativa em 2014 segundo o instituto.
		
		Iniciativas educacionais e de marketing sobre o t\'{o}pico come\c{c}am a ser produzidos em larga escala por grandes empresas e comunidades nas redes sociais relacionada �s Internet das Coisas como Linkedin e a plataforma de "networking" \_connect do Conselho de Estrat\'{e}gia Tecnol\'{o}gica do Reino Unido.
		
		O amadurecimento das plataformas de prototipa\c{c}\~{a}o (Arduino) e dos computadores baseados em SoC (Raspberry Pi, Intel Edison) permitiu aos entusiastas investir na cria\c{c}\~{a}o de pequenos projetos de forma acess\'{i}vel e com isso verificamos uma expans\~{a}o na quantidade de plataformas (Pachube e Thingspeak), padr\~{o}es de protocolos (6LoWPAN, Dash7, etc), Sistemas Operacionais (Contiki, TinyOS, etc) que est\~{a}o sendo desenvolvidas espec\'{i}ficamente para dar forma ao ecossistema da Internet das Coisas. � com foco nesse ecossistema que empresas especializadas em suas sub-\'{a}reas j\'{a} est\~{a}o se formando como a Mocano, especializada em seguran\c{c}a para IoT.
		
		\paragraph{A partir de 2015}
		Com a socializa\c{c}\~{a}o da internet, o aumento de dispositivos inteligentes conectados e a experi\^{e}ncia dos usu\'{a}rios com os impactos das novas tecnologias na vida das pessoas, era de se esperar que disrupturas ao redor da "Intenet das Coisas" acontecessem. De acordo com ~\cite{Mashal2015} atuamente est\~{a}o alta pesquisas relacionadas aos conceitos da Web das Coisas (WoT) focando na reutiliza\c{c}\~{a}o de padr\~{o}es abertos da internet atual na interoperabilidade entre dispositivos inteligentes ao compartilhar as informa\c{c}\~{o}es coletadas por sensores ub\'{i}quos e da Web Social das Coisas (SWoT) ressaltando o estudo de novas formas para integrar dispositivos inteligentes ("smart things") entre eles e com os humanos, mas n\~{a}o apenas como ponte entre a realidade e o virtual, mas fazendo parte do mundo real, pondendo atuar como tomador de decis\~{o}es em favor das pessoas se necess\'{a}rio. Essa abordagem futur\'{i}stica da SWoT, baseada em conceitos de tecnologias calmas e redes sociais, faz dela a mais promissora atualmente com a consolida\c{c}\~{a}o da Internet das Coisas, se tornando uma \'{a}rea de pesquisas atualmente ativa.
		
		
		%% Incluir parte sobre WIoT, Edge Computing, CoT, Objetos Virtuais
