\section{Contextualiza\c{c}\~ao} 
A iniciativa desse trabalho tem como principal motiva\c{c}\~{a}o, analisar os desafios tecnol\'{o}gicos atualmente conduzidos por uma crescente demanda para ades\~{a}o \`{a} modelos de agricultura com foco na produ\c{c}\~{a}o de alimentos mais saud\'{a}veis, mantendo um meio ambiente sustent\'{a}vel. Como base para a pesquisa, foram analisados relat\'{o}rios t\'{e}cnicos e estudos com direcionamento dos principais \'{o}rg\~{a}os governamentais de apoio e estat\'{i}stica sobre o meio ambiente para um correto planejamento na agricultura, conservando e otimizando a utiliza\c{c}\~{a}o de recursos naturais.

De acordo com os dados disponibilibilizados pela ONG \acronym{ONG}{Organiza\c{c}\~{a}o sem fins lucrativos} Global Footprint Network \cite{GFN2017} para o ano de 2016, no dia 8 de agosto a humanidade consumiu em tempo recorde o total de recursos sustent\'{a}veis previstos para um ano completo, mostrando a import\^{a}ncia que o controle sobre a utiliza\c{c}\~{a}o de recursos possui nos dias atuais. 

Ao observar atentamente aos indicadores brasileiros de 2015 pontuados no relat\'{o}rio do IBGE \acronym{IBGE}{Instituto Brasileiro de Geografia e estat\'{i}stica} sobre desenvolvimento sustent\'{a}vel \cite{IBGE94254}, vemos que o setor agr\'{i}cola brasileiro possui como foco principal a busca e obten\c{c}\~{a}o dos \'{i}ndices de produtividade agropastoril necess\'{a}rios para satisfazer \`{a}s demandas de mercado, deixando uma produ\c{c}\~{a}o sustent\'{a}vel em segundo plano. 

Na busca em alcan\c{c}ar esses \'{i}ndices, junto a um custo de produ\c{c}\~{a}o competitivo, mantendo um bom controle de pragas, doen\c{c}as e ervas daninhas que poderiam afetar sua produ\c{c}\~{a}o, agricultores acabam manipulando incorretamente a quantidade de recursos naturais, minerais, fertilizantes e defensivos qu\'{i}micos utilizados, criando um desequil\'{i}brio no meio ambiente, geralmente associados aos danos \`{a} biodiversidade, processos de eutrofiza\c{c}\~{a}o em rios e lagos, comprometimento da qualidade dos recursos h\'{i}dricos com a contamina\c{c}\~{a}o de aqu\'{i}feros e reservat\'{o}rios, exposi\c{c}\~{a}o do solo \`{a} mudan\c{c}as clim\'{a}ticas e acidifica\c{c}\~{a}o excessiva, emiss\~{a}o de gases associados ao efeito estufa e o grande potencial de intoxica\c{c}\~{a}o e agravos \`{a} sa\'{u}de das pessoas. 

Isto posto, um controle eficiente e eficaz sobre os n\'{i}veis de recursos empregados na produtividade agropastoril, permitiria ao produtor agricola um melhor acompanhamento dos resultados de seu manejo no plantio, uma agilidade maior na identifica\c{c}\~{a}o e adequa\c{c}\~{a}o de dist\'{u}rbios, al\'{e}m de garantir os melhores n\'{i}veis de produ\c{c}\~{a}o e sustentabilidade da \'{a}rea utilizada, reduzindo consider\'{a}velmente a intensidade dos impactos no meio ambiente e seus riscos \`{a} qualidade dos solos, fontes de \'{a}gua e do produto final consumido pelas pessoas.

Como essas an\'{a}lises e controles de sustentabilidade se tornam extremamente custosas e complexas aos produtores, principalmente quando manualmente aferidas, elas passam a ser muitas vezes menosprezada ou abandonadas pelo produtor, principalmente na agricultura familiar que geralmente possui recursos financeiros limitados. Por outro lado, a automatiza\c{c}\~{a}o dos processos com o uso de sensores e computadores para aferi\c{c}\~{a}o e processamento dos dados apresentavam-se como \'{o}timas solu\c{c}\~{o}es, por\'{e}m afora um custo igualmente caro e uma elevada complexidade de implanta\c{c}\~{a}o, havia um esfor\c{c}o extra e com custo igualmente elevado para sua manuten\c{c}\~{a}o, que necessitava pessoas qualificadas e conhecimento ficava restrito aos fabricantes da nova estrutura.

Portanto, \'{e} necessário que os modelos tecnol\'{o}gicos propostos considerem custos de avalia\c{c}\~{a}o do desempenho ambiental de produtos ao longo de todo o seu ciclo de vida utilizando metodologias de análise de ciclo de vida \cite{Frankl2000} existentes (ICVs \acronym{ICV}{invent\'{a}rios do Ciclo de Vida de produtos agr\'{i}colas e agroindustriais} ou ACVs \cite{34779} \cite{CICLODEVIDA2017} \acronym{ACV}{avalia\c{c}\~{a}o do Ciclo de Vida de produtos}), de forma que a análises dessas informa\c{c}\~{o}es se tornem menos onerosas para o processo como um todo, criando um rigoroso equil\'{i}brio entre produ\c{c}\~{a}o agr\'{i}cola e preserva\c{c}\~{a}o dos recursos naturais.

Em vários países, ICVs e ACVs são considerados para a formulação de novas pol\'{i}ticas p\'{u}blicas por ser uma metodologia com forte base cient\'{i}fica reconhecida internacionalmente e padronizada por normas ISO \cite{ISO14040-2006} \cite{ISO14044-2006}. Al\'{e}m do suporte \`{a}s pol\'{i}ticas p\'{u}blicas, esses modelos tamb\'{e}m permitem ao setor privado produzir seus produtos com as melhores pr\'{a}ticas cient\'{i}ficas de redu\c{c}\~{a}o dos impactos ao meio ambiente. No Brasil, os invent\'{a}rios nacionais são mantidos pelo IBICT \cite{PBACV2017} \acronym{IBICT}{Instituto Brasileiro de Informa\c{c}\~{a}o em Ci\^{e}ncia e Tecnologia} e podem ser verificados utilizando o banco nacional de invent\'{a}rios do ciclo de vida em \cite{SICV2017}.

O desenvolvimento atual dos processos de irriga\c{c}\~{a}o, por exemplo, depende de procedimentos tecnol\'{o}gicos e econ\^{o}micos para otimizar o uso da \'{a}gua, melhorar a efici\^{e}ncia de aplica\c{c}\~{a}o, proporcionar ganhos de produtividade baseados na resposta da cultura \`{a} aplica\c{c}\~{a}o de \'{a}gua e outros insumos, sem que comprometa a disponibilidade e qualidade do recurso.

A agricultura familiar nessas condi\c{c}\~{o}es passa por enormes dificuldades, uma vez que sem \'{a}gua \'{e} imposs\'{i}vel cultivar. Muitas comunidades de agricultores familiares est\~{a}o instaladas em regi\~{o}es pr\'{o}ximas de rios ou reservat\'{o}rios, onde a agricultura irrigada vem sendo cada vez mais difundida e sendo um atrativo a essas comunidades, que t\^{e}m se mobilizado no sentido de inserir-se no processo produtivo. \'{E} necess\'{a}rio, entretanto, capacitar esses novos irrigantes, bem como os que j\'{a} est\~{a}o estabelecidos, quer em projetos p\'{u}blicos, assentamentos ou em situa\c{c}\~{o}es particulares, para fazer uso adequado da \'{a}gua retirada de fontes de \'{a}gua cada vez mais reduzidas. Esta cartilha tem inten\c{c}\~{a}o de fornecer a agricultores familiares informa\c{c}\~{o}es e conhecimentos b\'{a}sicos de sistemas de irriga\c{c}\~{a}o, considerando seu uso voltado para conserva\c{c}\~{a}o de \'{a}gua, bem como informa\c{c}\~{o}es sobre como usar a \'{a}gua para irrigar as plantas de modo a manter o solo com umidade suficiente para uma produ\c{c}\~{a}o adequada com perdas m\'{i}nimas de \'{a}gua. 

