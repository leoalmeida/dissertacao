\chapter{Conclus�o}\label{cap:conclusao}
Concluindo, podemos ver que todas os paradigmas analisados possuem grande sinergia, por�m analisando mais atentamente, vemos que na verdade a grande semelhan�a entre elas tem como origem o fato de serem ramifica��es da Computa��o Ub�qua. Podemos resumir a  diferan�a entre os paradigmas por suas principais propostas, da seguinte forma: 

\begin{labeling}{Disappearing Computing}
	\item[Computa��o Ub�qua] Aprimorar dispositivos ao ponto que sua utiliza��o se tornar� impercept�vel.
	\item[Disappearing Computing] Definir tecnologias respons�veis pela transforma��o de computadores em objetos impercept�veis no ambiente.
	\item[Internet das Coisas] Unir dispositivos e sensores criando uma rede de dispositivos ub�quos.
	\item[Web das Coisas] Garantir que dispositivos ub�quos heterog�neos possam se localizar e interagir na internet de forma transparente.
	\item[Web Social das Coisas] Explorar redes sociais incluindo dispositivos ub�quos que consigam interpretar e serem interpretados corretamente entre si e com as pessoas.
\end{labeling}

Por fim, novos projetos, eventos, apresenta��es e not�cias sobre esses paradigmas s�o inclu�dos di�riamente na internet por pesquisadores e entusiastas no assunto. Ao avaliar seus pr�ximos passos, busque distinguir paradigmas e tecnologias que sendo pesquisadas no momento, identificando assim o caminho prov�vel para as novas oportunidades e desafios.