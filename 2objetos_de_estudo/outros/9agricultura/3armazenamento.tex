\section{Modelos agr\'{i}colas relacionados ao trabalho}
Considerando o dom\'{i}nio da agricultura familiar e da atividade  agropastoril, por exemplo, nota-se que o acompanhamento dos n\'{i}veis de nutrientes e irriga\c{c}\~{a}o do solo pode ser facilitado atrav\'{e}s da implanta\c{c}\~{a}o de diversos sensores capazes de coletar informa\c{c}\~{o}es e transmiti-las em rede para algum sistema de informa\c{c}\~{a}o. Com essas informa\c{c}\~{o}es, um agricultor pode realizar o planejamento das atividades de reposi\c{c}\~{a}o de nutrientes e irriga\c{c}\~{a}o do solo. Al\'{e}m disso, ele pode verificar quais regi\~{o}es de sua \'{a}rea de plantio apresentam maior desgaste e requerem reparos antecipadamente.

\subsection{An\~{a}lise de recursos naturais}
\subsection{Estrat�gias para armazenamento de recursos naturais}
\subsection{Sistemas de Irriga\c{c}\~{a}o}
\subsection{Sistemas para distribui\c{c}\~{a}o de outros recursos}
\subsection{Irriga\c{c}\~{a}o por micro-asper\c{c}\~{a}o}
\subsection{An\~{a}lise de Insetos}
	
\subsection{Formas de manipula\c{c}\~{a}o de fertilizantes e defens�vos}
	\cite{Santos2014}

\subsection{Identifica\c{c}\~{a}o e An\~{a}lise do meio ambiente}