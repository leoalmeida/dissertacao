\section{Identifica��o dos objetos na rede}
	Embora seja relativamente novo, o dom�nio de Internet das Coisas tem sido alvo de pesquisas h� algum tempo. Universidades, empresas e organiza��es t�m empregado esfor�os para definir, propor e implementar solu��es que prover�o suporte para essa nova �rea que tende a se popularizar nos pr�ximos anos.	
	
\subsection{Agrupamento de objetos por Federa��es}
	 A infraestrutura baseada em federa��o (IOT-A, 2011) consiste em uma arquitetura proposta com base em conceitos apresentados em (MCLEOD e HEIMBIGNER, 1980) e busca tratar a heterogeneidade de cena?rios e recursos sem exigir uma abordagem que force uma solu��o u?nica para a resolu��o de nomes. Nessa arquitetura, propo?e-se que cada n� represente um local que agregue diversos recursos e	

\subsection{Agrupamento de objetos utilizando RNS}
	Algumas abordagens, como o RNS, proposto em (TIAN et al, 20012), buscam manter a compatibilidade com os sistemas de Internet das Coisas j� existentes. Projetado para ser uma plataforma capaz de suportar sistemas de nomea��o e resolu��o distintos, o RNS busca n�o exigir altera��es significativas nos sistemas j� criados.


\section{Sinks e algoritmos de coleta de dados}

