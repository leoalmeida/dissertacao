\chapter{Experimentos e resultados}
Neste capítulo, é apresentada a metodologia utilizada para gerar os experimentos e os resultados obtidos. Para tanto, a validação do método proposto, nomeado NodePM, foi dividida em três etapas. Na primeira etapa, Seção 4.2, apresenta e discute as novidades/anomalias encontradas no consumo de energia dos equipamen- tos eletrônicos. Na segunda etapa, Seção 4.3, é realizado uma avaliação de desempenho tendo como parâmetro de comparação o Self-Organizing Novelty Detection (SONDE), des- crito na Subseção 4.1.2. Por fim, na terceira etapa, Seção 4.4, a eficiência em detectar anomalias no consumo de energia dos equipamentos eletrônicos é validada.


\section{Metodologia}

Segundo Jain (1991), cada sistema computacional possui a sua particularidade. Dessa forma, a avaliação de desempenho torna-se algo peculiar para cada sistema avaliado. Nesse sentido, não é comum usar a mesma metodologia de avaliação para sistemas com- putacionais diferentes. Para esta pesquisa três etapas compõem a avaliação realizada, sendo elas: (i) a definição da base de dados; (ii) a escolha do baseline para o método; e (iii) o ambiente monitorado. A seguir, cada uma dessas etapas é descrita.


\subsection{Base de dados}


O método proposto utilizou um conjunto de dados (dataset), contendo informações reais sobre a interação do usuário com o equipamento eletrônico. Essa base de dados possui quatro atributos, sendo eles: (i) identificador do equipamento atribuído pelo sensor; (ii) potência em Watt; (iii) a data de utilização do equipamento; (iv) e o tempo de uso do equipamento naquele momento.


\subsection{O baseline para o algoritmo proposto}
Para realizar uma análise de desempenho do NodePM, como será feito na Seção 4.3, foi necessário obter um baseline. Para tanto, o baseline foi o Self-Organizing Novelty Detection (SONDE) (Albertini & Mello, 2007). A SONDE é uma rede neural que adapta incremen- talmente sua estrutura de conhecimento (neurônio), com o objetivo de detectar novidades em ambientes dinâmicos. Para isso, a SONDE classifica em um mesmo neurônio padrões de entradas similares. Quando nenhum neurônio é capaz de classificar um padrão de entrada, um novo neurônio é criado indicando uma novidade no ambiente.


\subsection{Descrição do ambiente monitorado}
Para produzir resultados com boa precisão, foi criado um ambiente real para monitorar o consumo de energia dos equipamentos eletrônicos em uma residência. Nesse cenário, os equipamentos se localizavam em cômodos diferentes e, por conveniência, o gateway Arduino estava junto do roteador para comunicar com a Internet. O monitoramento do consumo de energia dos equipamentos eletrônicos, sumarizados na Tabela 4.1, foram feitas durante os anos de 2012 (Julho, Agosto, Setembro, Novembro e Dezembro) e 2013 (Janeiro, Fevereiro, Junho e Julho) com período de tempo variável.


\section{Experimentos de detecção de novidades com o NodePM}
Para alcançar um dos objetivos deste trabalho, foi necessário investigar como as novi- dades ocorrem. Para tanto, utilizou-se as partições do conjunto de dados para a realização dos experimentos. Com isto, foi possível identificar dois tipos de novidades: qualitativa e quantitativa.

\begin{itemize}
	\item{Tipo 1} Ocorre quando o equipamento eletrônico tem uma mudança abrupta no seu comportamento padrão, ou seja, ocorreu uma troca não habitual no seu estado, passando do estado x para o estado y, situação essa que não era esperado. (qualita- tiva)
	\item{Tipo 2} Acontece quando o equipamento eletrônico começa a consumir mais energia que o esperado durante um determinado período, ou seja, passando do estado usual do usuário. (quantitativa)
\end{itemize}


A seguir alguns experimentos são discutidos, apresentando como acontece a novi- dade.


\section{Avaliação de desempenho do NodePM com a SONDE}
Esta seção avalia o desempenho do NodePM, considerando as seguintes medidas de de- sempenho: sensibilidade (conhecida por alguns autores como taxa de detecção), precisão, especificidade e acurácia. Essas medidas, apresentadas a seguir, são calculadas a partir de uma matriz de confusão ilustrada na Figura 4.4, no qual avalia os resultados com base nas perdas causadas.